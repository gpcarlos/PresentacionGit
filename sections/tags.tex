\begin{frame}{Etiquetado - Creación y eliminación}
  \begin{columns}[T,onlytextwidth]
    \column{0.6\textwidth}
    \begin{itemize}
      \item \alert{Etiquetas Ligeras} \\
        Puntero a commit
        \mint{console}|  $ git tag <nombre-tag>|
      \item \alert{Etiquetas Anotadas} \\
        Puntero, mensaje, etiquetador...
        \mint{console}|  $ git tag -a <nombre-tag> -m 'msj'|
    \end{itemize}
    \column{0.4\textwidth}
      \begin{tikzpicture}
  \gitDAG[grow up sep = 1.5em]{
    H -- I -- J -- K -- L
  };
  % Tag reference
  \gittag
    [v0p9]       % node name
    {v0.9}       % node text
    {left=of H} % node placement
    {H}          % target
  \gittag
    [v1p0]       % node name
    {v1.0}       % node text
    {left=of L} % node placement
    {L}          % target
  % Remote branch
  \gitremotebranch
    [origmaster]    % node name
    {\tiny origin/master} % node text
    {right=of J}    % node placement
    {J}             % target
  % Branch
  \gitbranch
    {master}      % node name and text
    {right=of L} % node placement
    {L}          % target
  % HDAD reference
  \gitHEAD
    {right=of master} % node placement
    {master}          % target
\end{tikzpicture}

  \end{columns}
\end{frame}

\begin{frame}{Etiquetado - Creación y eliminación}
  \begin{columns}[T,onlytextwidth]
    \column{0.6\textwidth}
    \begin{itemize}
      \item \alert{Etiquetado Tardío} \\
      \mint{console}|  $ git tag -a <nombre-tag> <SHA-1>|
      \item \alert{Eliminar etiqueta} \warnning \\
      \mint{console}|  $ git tag -d <nombre-tag>|
    \end{itemize}
    \column{0.4\textwidth}
      \begin{tikzpicture}
  \gitDAG[grow up sep = 1.5em]{
    H -- I -- J -- K -- L
  };
  % Tag reference
  \gittag
    [v0p9]       % node name
    {v0.9}       % node text
    {left=of H} % node placement
    {H}          % target
  \gittag
    [v1p0]       % node name
    {v1.0}       % node text
    {left=of L} % node placement
    {L}          % target
  % Remote branch
  \gitremotebranch
    [origmaster]    % node name
    {\tiny origin/master} % node text
    {right=of J}    % node placement
    {J}             % target
  % Branch
  \gitbranch
    {master}      % node name and text
    {right=of L} % node placement
    {L}          % target
  % HDAD reference
  \gitHEAD
    {right=of master} % node placement
    {master}          % target
\end{tikzpicture}

  \end{columns}
\end{frame}

\begin{frame}{Etiquetado - Listado}
  \begin{columns}[T,onlytextwidth]
    \column{0.6\textwidth}
    \begin{itemize}
      \item \alert{Todas}
        \mint{console}|  $ git tag|
      \item \alert{Por patrón}
        \mint{console}|  $ git tag -l 'v1*'|
      \item \alert{Específica}
        \mint{console}|  $ git show <nombre-tag>|
    \end{itemize}
    \column{0.4\textwidth}
      \begin{tikzpicture}
  \gitDAG[grow up sep = 1.5em]{
    H -- I -- J -- K -- L
  };
  % Tag reference
  \gittag
    [v0p9]       % node name
    {v0.9}       % node text
    {left=of H} % node placement
    {H}          % target
  \gittag
    [v1p0]       % node name
    {v1.0}       % node text
    {left=of L} % node placement
    {L}          % target
  % Remote branch
  \gitremotebranch
    [origmaster]    % node name
    {\tiny origin/master} % node text
    {right=of J}    % node placement
    {J}             % target
  % Branch
  \gitbranch
    {master}      % node name and text
    {right=of L} % node placement
    {L}          % target
  % HDAD reference
  \gitHEAD
    {right=of master} % node placement
    {master}          % target
\end{tikzpicture}

  \end{columns}
\end{frame}

\begin{frame}[fragile]{Etiquetado - Compartir}
  \begin{columns}[T,onlytextwidth]
    \column{0.6\textwidth}
    \texttt{git push} no transfiere etiquetas a remotos
    \begin{itemize}
      \item \alert{Todas}
        \mint{console}|  $ git push origin --tags|
      \item \alert{Específica}
        \mint{console}|  $ git push origin <nombre-tag>|
    \end{itemize}
    \column{0.4\textwidth}
      \begin{tikzpicture}
  \gitDAG[grow up sep = 1.5em]{
    H -- I -- J -- K -- L
  };
  % Tag reference
  \gittag
    [v0p9]       % node name
    {v0.9}       % node text
    {left=of H} % node placement
    {H}          % target
  \gittag
    [v1p0]       % node name
    {v1.0}       % node text
    {left=of L} % node placement
    {L}          % target
  % Remote branch
  \gitremotebranch
    [origmaster]    % node name
    {\tiny origin/master} % node text
    {right=of J}    % node placement
    {J}             % target
  % Branch
  \gitbranch
    {master}      % node name and text
    {right=of L} % node placement
    {L}          % target
  % HDAD reference
  \gitHEAD
    {right=of master} % node placement
    {master}          % target
\end{tikzpicture}

  \end{columns}
\end{frame}

\begin{frame}[fragile]{Etiquetado - Sacar una etiqueta}
  \begin{columns}[T,onlytextwidth]
    \column{0.6\textwidth}
    Para colocar en el directorio de trabajo una versión, es necesario crear una nueva rama en esa etiqueta.
    \mint{console}|  $ git checkout -b <n-rama> <n-tag>|
    \uncover<2>{
    \vspace{0.5cm}
    \begin{exampleblock}{Ejemplo:}
      \mint{console}|  $ git checkout -b version0p9 v0.9|
    \end{exampleblock}}
    \column{0.4\textwidth}
      \only<1>{\begin{tikzpicture}
  \gitDAG[grow up sep = 1.5em]{
    H -- I -- J -- K -- L
  };
  % Tag reference
  \gittag
    [v0p9]       % node name
    {v0.9}       % node text
    {left=of H} % node placement
    {H}          % target
  \gittag
    [v1p0]       % node name
    {v1.0}       % node text
    {left=of L} % node placement
    {L}          % target
  % Remote branch
  \gitremotebranch
    [origmaster]    % node name
    {\tiny origin/master} % node text
    {right=of J}    % node placement
    {J}             % target
  % Branch
  \gitbranch
    {master}      % node name and text
    {right=of L} % node placement
    {L}          % target
  % HDAD reference
  \gitHEAD
    {right=of master} % node placement
    {master}          % target
\end{tikzpicture}
}
      \only<2>{\begin{tikzpicture}
  \gitDAG[grow up sep = 1.5em]{
    H -- I -- J -- K -- L
  };
  % Tag reference
  \gittag
    [v0p9]       % node name
    {v0.9}       % node text
    {left=of H} % node placement
    {H}          % target
  \gittag
    [v1p0]       % node name
    {v1.0}       % node text
    {left=of L} % node placement
    {L}          % target
  % Remote branch
  \gitremotebranch
    [origmaster]    % node name
    {\tiny origin/master} % node text
    {right=of J}    % node placement
    {J}             % target
  % Branch
  \gitbranch
    {master}      % node name and text
    {right=of L} % node placement
    {L}          % target
  \gitbranch
    {version0p9}      % node name and text
    {right=of H}% node placement
    {H}          % target
  % HDAD reference
  \gitHEAD
    {above=of version0p9} % node placement
    {version0p9}          % target
\end{tikzpicture}
}
  \end{columns}
\end{frame}
